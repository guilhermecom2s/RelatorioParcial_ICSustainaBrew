\chapter{Planejamento de atividades}

As próximas etapas do projeto estão orientadas à consolidação dos dados coletados, à realização de análises comparativas entre microcervejarias brasileiras e alemãs e ao aprofundamento da implementação prática da transformação digital no contexto do processo cervejeiro. As atividades planejadas visam dar continuidade ao trabalho já desenvolvido.

Inicialmente, à medida que for alcançado um número satisfatório de respostas provenientes das microcervejarias alemãs, será finalizada a etapa de tratamento de dados do questionário. Esse processo incluirá a padronização, organização e verificação de consistência das respostas, de modo a garantir a comparabilidade entre os conjuntos de dados obtidos no Brasil e na Alemanha.

Concluído o tratamento dos dados, será realizada uma análise exploratória com o objetivo de caracterizar os perfis das microcervejarias de ambos os países. Essa análise buscará identificar padrões, tendências e principais características relacionadas aos processos produtivos, aos indicadores de desempenho acompanhados e ao nível de adoção de tecnologias digitais. A análise exploratória permitirá uma compreensão inicial das semelhanças e diferenças entre os contextos analisados, servindo de base para investigações estatísticas mais aprofundadas.

Na sequência, será conduzida uma análise estatística comparativa entre as microcervejarias brasileiras e alemãs. Essa etapa terá como foco a identificação de diferenças e similaridades relevantes entre os dois grupos, considerando aspectos organizacionais, produtivos e tecnológicos. Em particular, a análise buscará compreender como práticas associadas à transformação digital se manifestam em cada contexto, bem como identificar oportunidades de aprimoramento e transferência de boas práticas entre as empresas de ambos os países.

Paralelamente às análises de dados, será finalizada a implementação do piloto de transformação digital iniciado na cervejaria do laboratório do campus. Essa etapa incluirá o desenvolvimento de uma interface que permita ao usuário acompanhar, de forma segura e intuitiva, as diferentes etapas do processo produtivo cervejeiro. Serão elaborados painéis específicos de monitoramento no Grafana para cada etapa do processo, possibilitando o acompanhamento de variáveis e indicadores relevantes.

%Por fim, espera-se que a consolidação dessas atividades permita não apenas a conclusão do projeto de iniciação científica, mas também a construção de uma base sólida para trabalhos futuros, incluindo a elaboração de artigos científicos e a ampliação das análises sobre transformação digital no setor cervejeiro, com foco em pequenas e médias empresas.
