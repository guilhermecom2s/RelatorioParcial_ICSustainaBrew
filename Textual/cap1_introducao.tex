\chapter{Introdução}


%Seções sugeridas:

%\section{Objetivos}

%\section{Descrição de composição e funcionamento do CPA/CPE}

%\subsection{Estrutura de Gestão}

%\subsection{Parcerias}

%\subsubsection{Órgãos Públicos}

%\subsubsection{Empresas}

\section{Transformação digital e Indústria 4.0}

A transformação digital tem sido amplamente discutida na literatura como um processo organizacional abrangente, que vai além da simples adoção de tecnologias digitais e envolve mudanças profundas em estratégias, estruturas, processos e modelos de negócio. Segundo Vial \cite{vial2021understanding}, a transformação digital pode ser compreendida como um fenômeno multidimensional, no qual tecnologias digitais atuam como catalisadoras de mudanças organizacionais, afetando simultaneamente a criação de valor, a proposta de valor e os mecanismos de captura de valor das organizações.

No contexto industrial, a transformação digital está fortemente associada ao paradigma da Indústria~4.0, entendido como a quarta revolução industrial. Ghobakhloo \cite{ghobakhloo2020industry} descreve a Indústria~4.0 como um processo de digitalização sistêmica da manufatura e das cadeias de valor, fundamentado na integração de tecnologias digitais, sistemas físicos e fluxos de informação. Diferentemente das revoluções industriais anteriores, a Indústria~4.0 caracteriza-se pela conectividade contínua entre ativos produtivos, sistemas computacionais e atores humanos, possibilitando tomada de decisão descentralizada e orientada por dados.

Sob essa perspectiva, a Indústria~4.0 não se limita à automação de processos, mas envolve a reconfiguração das arquiteturas produtivas e organizacionais. Ghobakhloo \cite{ghobakhloo2020industry} destaca princípios fundamentais associados a esse paradigma, como interoperabilidade, descentralização, integração horizontal e vertical, virtualização e capacidade de operação em tempo real. Esses princípios sustentam o desenvolvimento de sistemas produtivos inteligentes, capazes de monitorar e ajustar processos de forma dinâmica ao longo de todo o ciclo de vida do produto.

\section{Arquiteturas digitais e transformação digital}

A viabilização prática da transformação digital e da Indústria~4.0 depende diretamente das arquiteturas computacionais utilizadas para coleta, processamento e armazenamento de dados. Nesse contexto, Sestino et al.~\cite{sestino2020internet} ressaltam o papel central das tecnologias digitais habilitadoras, incluindo computação em nuvem, Internet das Coisas e análise de dados, como elementos estruturantes dos novos modelos de negócio digitais.

A computação em nuvem destaca-se por permitir o acesso remoto e escalável a recursos computacionais, reduzindo barreiras de entrada tecnológicas e financeiras, especialmente para pequenas e médias empresas. Conforme discutido por Sestino et al.~\cite{sestino2020internet}, arquiteturas baseadas em nuvem favorecem a centralização de dados, a integração de sistemas e o desenvolvimento de serviços digitais orientados a dados, contribuindo para a inovação em modelos de negócio.

Entretanto, Ghobakhloo \cite{ghobakhloo2020industry} aponta que a digitalização industrial impõe requisitos específicos relacionados à latência, confiabilidade e operação em tempo real, que nem sempre são plenamente atendidos por arquiteturas exclusivamente centralizadas. Embora o autor não trate explicitamente do conceito de computação de borda, sua análise evidencia a necessidade de arquiteturas distribuídas e descentralizadas, nas quais decisões operacionais possam ser tomadas localmente, em alinhamento com o princípio de descentralização da Indústria~4.0.

Dessa forma, observa-se que a literatura associada à transformação digital industrial sugere a adoção de arquiteturas híbridas, combinando centralização de dados e processamento distribuído, como estratégia para equilibrar eficiência computacional, confiabilidade operacional e flexibilidade organizacional.

\section{Transformação digital em pequenas e médias empresas}

Grande parte dos estudos iniciais sobre Indústria~4.0 concentrou-se em grandes corporações industriais; entretanto, trabalhos mais recentes reconhecem que pequenas e médias empresas também podem se beneficiar da transformação digital, desde que adotem abordagens compatíveis com suas limitações estruturais. Schallmo et al.~\cite{schallmo2017digital} enfatizam que a transformação digital deve ser entendida como um processo estratégico e incremental, no qual organizações de menor porte podem explorar tecnologias digitais de forma progressiva, alinhando inovação tecnológica e viabilidade econômica.

Nesse sentido, a transformação digital em pequenas empresas tende a ocorrer por meio da adoção de soluções modulares, uso intensivo de software e exploração de tecnologias digitais de menor custo, como plataformas baseadas em nuvem, algoritmos de aprendizado de máquina e sistemas de monitoramento digital. Sestino et al.~\cite{sestino2020internet} reforçam que essas tecnologias não apenas suportam a digitalização de processos existentes, mas também possibilitam a criação de novos modelos de negócio, mesmo em contextos organizacionais com recursos limitados.

Aplicado ao contexto industrial analisado neste projeto, esse conjunto de abordagens sugere que a transformação digital pode ser implementada de forma gradual e adaptativa, respeitando as especificidades produtivas e organizacionais de pequenas unidades industriais, como microcervejarias, sem a necessidade de investimentos elevados em automação pesada, mas mesmo assim, gerando valor na cadeia produtiva.


